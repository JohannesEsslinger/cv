\documentclass[11pt, a4paper]{moderncv}

\newcommand{\en}[1]{#1}
\newcommand{\de}[1]{}

\en{\usepackage[english]{babel}}
\de{\usepackage[ngerman]{babel}}

\usepackage[utf8]{inputenc}
\usepackage[left=1.7cm, right=1.7cm, top=1.0cm, bottom=1.0cm, scale=0.8]{geometry}
% \usepackage{fontawesome}

% source: https://tex.stackexchange.com/a/294990/271172
\newcommand{\ExternalLink}{%
    \tikz[x=1.2ex, y=1.2ex, baseline=-0.05ex]{% 
        \begin{scope}[x=1ex, y=1ex]
            \clip (-0.1,-0.1) 
                --++ (-0, 1.2) 
                --++ (0.6, 0) 
                --++ (0, -0.6) 
                --++ (0.6, 0) 
                --++ (0, -1);
            \path[draw, 
                line width = 0.5, 
                rounded corners=0.5] 
                (0,0) rectangle (1,1);
        \end{scope}
        \path[draw, line width = 0.5] (0.5, 0.5) 
            -- (1, 1);
        \path[draw, line width = 0.5] (0.6, 1) 
            -- (1, 1) -- (1, 0.6);
        }
    }
\renewcommand{\link}[2]{\href{#1}{#2 \ExternalLink}}


\newcommand{\quotes}[1]{\en{``#1''}\de{\glqq{}#1\grqq{}}}
\renewcommand*{\namefont}{\fontsize{20}{22}\mdseries\upshape}

\moderncvtheme[blue]{classic}
\moderncvicons{awesome}
 
\name{Pavel}{Zwerschke}
\title{\en{CV}\de{Lebenslauf}}
% \address{}{}
% \phone[mobile]{}
\email{pavelzw@gmail.com}
\social[github][github.com/pavelzw]{pavelzw}
\social[linkedin][linkedin.com/in/pavel-zwerschke-384408204]{Pavel Zwerschke}
% \photo[2.5cm]{images/photo}
 
\begin{document}
\vspace*{-10mm}
\makecvtitle
\vspace*{-13mm}

\section{\en{Personal}\de{Persönliche Daten}}
\cvline{\en{born}\de{geboren}}{12.04.2001, Stuttgart}

\section{\en{Education}\de{Ausbildung}}
%\cventry{2010-2018}{Gymnasium}{Königin-Olga-Stift}{Stuttgart}{}{Abitur: 1,4}
\cventry{2021-\en{today}\de{heute}}
        {\en{Karlsruhe Institute of Technology (KIT)}\de{Karlsruher Institut für Technologie (KIT)}}
        {M. Sc. \en{Mathematics}\de{Mathematik}}
        {}
        {}
        {\en{German grade average: 1.0}\de{Durchschnittsnote: 1,0}}
\cventry{2021-\en{today}\de{heute}}
        {\en{Karlsruhe Institute of Technology (KIT)}\de{Karlsruher Institut für Technologie (KIT)}}
        {M. Sc. \en{Computer Science}\de{Informatik}}
        {}
        {}
        {\en{German grade average: 1.0}\de{Durchschnittsnote: 1,0}}
\cventry{2018-2021}
        {\en{Karlsruhe Institute of Technology (KIT)}\de{Karlsruher Institut für Technologie (KIT)}}
        {B. Sc. \en{Mathematics}\de{Mathematik}}
        {}
        {}
        {\en{German grade average: 1.0 (with distinction)\\
        Thesis: \quotes{Nonparametric Distributional Regression Models for Probabilistic Energy Forecasting}}
        \de{Durchschnittsnote: 1,0 (mit Auszeichnung)\\
        Thesis: \quotes{Nonparametric Distributional Regression Models for Probabilistic Energy Forecasting}}}
\cventry{2019-2021}
        {\en{Karlsruhe Institute of Technology (KIT)}\de{Karlsruher Institut für Technologie (KIT)}}
        {B. Sc. \en{Computer Science}\de{Informatik}}
        {}
        {}
        {\en{German grade average: 1.1 (with distinction)}\de{Durchschnittsnote: 1,1 (mit Auszeichnung)}}

%\section{Praktika/Workshops/Studienreisen}
%\cvitem{08.2014}{Schülersprachreise in Exeter}
%\cvitem{09.2014}{TSCHIRA-Jugendakademie -- Forscherkurse für junge WissensSchaffer: \quotes{Aufbaukurs: Was ist Leben?} (Uni Heidelberg)}
%\cvitem{10.2014}{Schüleraustausch Russland, Samara}
%\cvitem{2014-2015}{Traineeprogramm Pädagogik für eine Waldheim-Jugendgruppenleiterausbildung} %todo hbox
%\cvitem{02.2016}{Berufsorientierung Praktikum, Robert Bosch GmbH, Stuttgart}
%\cvitem{02.2017}{Workshop \quotes{Rechnergestützte Produktentwicklung unter Einsatz eines 3D-Druckers} (Uni Stuttgart)}
%\cvitem{02.2017}{Diverse Workshops in Mathematik, Informatik, Physik an der Universität Stuttgart}
%\cvitem{04.2017}{Schüleraustausch USA, Salem (Oregon)}
%\cvitem{07.2017}{1. Platz beim Science Quiz (Kepler Seminar)}
%\cvitem{09.2017}{Schnupperstudium Informatik am KIT}
%\cvitem{02.2018}{Teilnahme an der Bundesrunde der 49. Internationalen Physikolympiade}

\section{\en{Experience}\de{Berufserfahrung}}
% \cvitem{2015-2018}{Jugendteamleiter im Waldheim Johannes und Waldheim Waldebene Ost, Stuttgart}
\cventry{04.2022-\en{today}\de{heute}}
        {QuantCo}
        {\en{Working Student}\de{Werkstudent}}
        {}
        {}
        {\en{I Implemented a more efficient algorithm to store sklearn models. Results in 3x disk space savings.}
        \de{Ich habe einen effizienteren Algorithmus für das Speichern von sklearn-Modellen entwickelt mit dreifacher Speichereffizienz.}}
\cventry{2021-2022}
        {\en{Karlsruhe Institute of Technology (KIT)}\de{Karlsruher Institut für Technologie (KIT)}}
        {\en{Scientific Assistant}\de{Wissenschaftliche Hilfskraft}}
        {}
        {}
        {\en{I was participating in the \quotes{S2S AI Challenge}, a challenge to improve sub-seasonal to seasonal weather predictions using artificial intelligence.}
        \de{Ich habe in der \quotes{S2S AI Challenge} mitgemacht, einem Wettbewerb zum Verbessern der sub-seasonal to seasonal Wettervorhersagen mithilfe von künstlicher Intelligenz.}}
\cventry{2019-2021}
        {\en{Karlsruhe Institute of Technology (KIT)}\de{Karlsruher Institut für Technologie (KIT)}}
        {\en{Tutor}\de{Tutor}}
        {}
        {}
        {\en{I gave tutorials in multiple subjects such as \quotes{Basic Notions of Computer Science}, \quotes{Software Engineering}, \quotes{Linear Algebra} and \quotes{Introduction to Probability Theory}.}
        \de{Ich habe Tutorien in mehreren Modulen gehalten: \quotes{Grundbegriffe der Informatik}, \quotes{Softwaretechnik}, \quotes{Lineare Algebra 1} und \quotes{Einführung in die Stochastik}.}}
\cvitem{2016-2018}{\en{Tutor in mathematics, physics and french}
        \de{Nachhilfelehrer für Mathematik, Physik und Französisch}}


\section{\en{Programming Skills}\de{Programmierkenntnisse}}
\cvcomputer{\en{Languages}\de{Sprachen}}{Python, Java, Bash, SQL, LaTeX}{\en{Technologies}\de{Technologien}}{Linux, NumPy, pandas, scikit-learn, PyTorch, Docker, Git}

\section{\en{Open Source Projects}\de{Open Source Projekte}}
\cvlistitem{\link{https://github.com/pavelzw/calibre-kindle-comics}{\textbf{calibre-kindle-comics}}: 
            \en{A calibre plugin that converts .cbz and .cbr files into a readable format for Kindle viewer.}
            \de{Ein calibre Plugin, welches .cbz und .cbr Dateien in ein lesbares Format für Kindle Geräte umwandelt.}}
\cvlistitem{\link{https://github.com/pavelzw/7circles}{\textbf{7circles}}: 
            \en{An educational video created with manim about the Seven-Circles-Theorem. The video is available on \link{https://youtu.be/flR3e5Cc2G4}{YouTube}.}
            \de{Ein mit manim animiertes Video über den Sieben-Kreise-Satz. Das Video ist auf \link{https://youtu.be/flR3e5Cc2G4}{YouTube} verfügbar.}}
\cvlistitem{\textbf{\en{Further contributions}\de{Weitere Beiträge}}: 
            Microsoft PowerToys (\link{https://github.com/microsoft/PowerToys/pulls?q=is\%3Apr+author\%3Apavelzw+}
            {\en{Added localized number formatting support}\de{lokalisierte Zahlenformatierung hinzugefügt}})}

\section{\en{Languages}\de{Sprachen}}
% use \cvcomputer for multi column descriptions
\cvcomputer{\en{German}\de{Deutsch}}{\en{Native}\de{Muttersprache}}
           {\en{English}\de{Englisch}}{\en{Fluent}\de{Fließend}}
\cvcomputer{\en{French}\de{Französisch}}{\en{Advanced}\de{Sehr gute Kenntnisse}}
           {\en{Russian}\de{Russisch}}{\en{Good}\de{Gute Kenntnisse}}
\cvcomputer{\en{Dutch}\de{Niederländisch}}{\en{Basic}\de{Grundlegende Kenntnisse}}
           {\en{Japanese}\de{Japanisch}}{\en{Basic}\de{Grundlegende Kenntnisse}}

\section{\en{Hobbys}\de{Hobbys, Interessen}}
\cvlistdoubleitem{\en{Piano}\de{Klavier}}{Tennis}
\cvlistdoubleitem{\en{Dancing}\de{Tanzen}}{}

\emptysection{}\closesection
\vfill
Pavel Zwerschke\\
Karlsruhe, \today
\end{document}